
% Preamble
\documentclass[14pt]{article}
%encoding
%--------------------------------------
\usepackage[T1]{fontenc}
\usepackage[utf8]{inputenc}
%--------------------------------------

%Portuguese-specific commands
%--------------------------------------
\usepackage[portuguese]{babel}
%--------------------------------------

%Hyphenation rules
%--------------------------------------
\usepackage{hyphenat}
\hyphenation{mate-mática recu-perar}
\usepackage{fancyhdr}

\pagestyle{fancy}
\fancyhf{}
\fancyhead[LE,RO]{AED}
\fancyhead[RE,LO]{AirRoutes}
\fancyfoot[RE,CO]{António Vidais & Tiago Leite}
\fancyfoot[LE,RO]{\thepage}
%--------------------------------------
\title{AirRoutes - Algoritmos e Estruturas de Dados}
\date{Dezembro de 2020}
\author{António Vidais \\96162 \and Tiago Leite \\96332}
% Document
\begin{document}
    \maketitle
    \section{Descrição do Problema}
    Este projeto, criado no âmbito da Unidade Curricular de Algoritmos e Estruturas de Dados, aborda o problema de
    encontrar um conjunto mínimo de rotas que garantam a existência de um caminho entre cada par de aeroportos,
    sem qualquer redundância. O Programa que este projeto visa produzir deve não
    só produzir uma rede mínima de rotas, mas sim a rede que garante todos os destinos já existentes, com o menor
    custo.
    Uma ferramenta como esta pode ser necessária, em termos práticos,quando uma companhia aérea se encontra num cenário
    de contenção de custos.
    Numa situação desta natureza, poderá ser necessário reduzir o seu conjunto de rotas à rede que
    garante que todos os aeroportos nela podem ser utilizados, com o menor custo.

    O Projeto que este relatório descreve foi desenvolvido na linguagem C e, para o seu correto funcionamento deve
    receber as informações de todas as rotas existentes através de um ficheiro de texto.
    Este deve conter uma ou mais redes de aeroportos, cada uma devidamente identificada com um cabeçalho contendo o
    número de aeroportos, o número total de rotas existentes e a variante que se pertende obter.
    Terminada a execução, o programa produz um segundo ficheiro de texto que contêm o conjunto mínimo de menor custo
    para cada rede fornecida, também identificados pelo cabeçalho fornecido, com a adição de algumas informações úteis,
    como o número de rotas mantidas e o custo total da rede

    \section{Abordagem do Problema}
    Para resolver este problema, foi necessário criar uma estrutura capaz de guardar uma representação do grafo formado
    pelo conjunto de rotas fornecido.
    A escolha desta estrutura requer uma análise cuidada ao tipo de grafo (denso ou esparço)e aos algoritmos que se
    pretende utilizar.
    Uma má escolha poderia levar a tempos de execução muito maiores que os pretendidos e maior utilização de memória,
    devido ao aumento da complexidade.
    
    \section{Arquitetura do Projeto}

\end{document}